\section{Conclusions}

In this section, we summarize the key contributions of our
work, acknowledge limitations, and outline possible 
directions for future development. We conclude with a 
brief reflection on the broader impact of our approach.

\subsection{Summary}

We introduced \texttt{tf::tree}, a spatial acceleration 
structure built for performance, simplicity, and flexibility 
in geometric computing. By combining data-oriented layouts, 
in-place partitioning with state-of-the-art selection algorithms
(ensuring $\mathcal{O}(n\log(n))$ average construction), and 
stack-based traversal schemes, \texttt{tf::tree} achieves 
high efficiency across construction, collision detection, 
(self) intersection and nearness queries.

Although \texttt{tf::tree} is structurally static, its 
query APIs support geometry undergoing linear 
transformations through user-defined callables. This 
enables dynamic queries over moving or deforming models 
without rebuilding the tree.

Empirical evaluations show that \texttt{tf::tree} achieves
near-linear scalability in construction and delivers real-time
query performance on models with millions of primitives. Notably,
in nearness queries, our Top-K sorted stack traversal provides a
practical speedup of up to $3\times$ over traditional heap-based
methods by trading an increase in AABB inspections for superior
cache performance and execution efficiency, demonstrating the
effectiveness of hardware-aware algorithmic design.

\subsection{Limitations and Future Work}

While \texttt{tf::tree} supports dynamic queries through
transformed callables, its internal structure remains fixed
after construction. Extending the system to support
incremental updates or fully dynamic scenes—particularly
those involving free-form deformation—remains a compelling
direction for future work. Although we employ such
capabilities in practice, the procedure has not yet been
formalized or exposed through the public API.

At present, search queries such as collision detection and 
intersection are parallelized, whereas nearness queries 
remain sequential. Parallelizing nearness queries presents 
an opportunity to further improve performance, especially 
on high-core-count systems. This is a natural extension of 
our current design and will be addressed in future work.

\subsection{Final Remarks}

\texttt{tf::tree} offers a practical and performant 
foundation for spatial queries in geometry processing. Its 
modular, open-source design is intended to be easily 
adapted and extended. We hope it proves valuable both as 
a production tool and as a reference for building fast and 
clean spatial structures.

