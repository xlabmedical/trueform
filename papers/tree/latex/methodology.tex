\section{Methodology}

This section presents the core components of \texttt{tf::tree}:
its data structures, construction algorithm, and query mechanisms.

We begin by describing the tree layout, designed for memory locality
and structural simplicity. We then introduce a construction procedure
based on in-place partitioning using state-of-the-art selection algorithms
Finally, we outline how the tree supports collision, (self) intersection,
and nearest-neighbor queries using stack-based traversals that
outperform traditional priority queues in practice.


\subsection{Tree Data Structure}

For performance reasons, we must consider about how the processor accesses memory \cite{data-oriented-design}.
That means paying attention to temporal and spatial locality. In practice, this leads to
two important design rules: first, all our data should be stored in contiguous arrays;
second, elements that are accessed together—like the children of a node—should also be
laid out next to each other in memory.
These simple but strict constraints shape the structure of our tree and how we represent
its nodes, primitives, and their relationships.


We now define the core data structures underlying our tree representation.


\begin{definition}[ID range]\label{def:ids}
	For a collection of primitives $P_i$, indexed by $i$,
	an ID range is an array $I$ containing a permutation
	of the sequence $[0 : N)$, where $N$ is the number of
	primitives. An ID sub-range $I[i:j]$ is a sub-array of
	$I$, containing elements $[I_i, I_{i+1}, \dots, I_{j-1}]$.
	The sub-range $I[i:j]$ may also be denoted by $(i, j - i)\subseteq I$.
\end{definition}


\begin{definition}[Node]\label{def:node}
	A node $N_i$ is part of an array of nodes $N$,
	indexed by $i$. It consists of an axis-aligned bounding box
	(\emph{aabb}) and associated metadata.

	A node is classified as an \emph{inner node} if its
	children are other nodes $N_c$, and as a \emph{leaf node}
	if its children are primitives $P_k$.

	The \emph{aabb} of $N_i$ encloses all primitives in its
	subtree. The metadata includes:
	\begin{itemize}
		\item an \emph{axis}, indicating the dimension along
		      which the node was partitioned (non-negative for
		      inner nodes, negative for leaves), and
		\item a \emph{data} range, given by an offset $o$ and
		      size $s$, indexing into its children.
	\end{itemize}

	For leaf nodes, this range refers to an ID sub-range
	$(o, s)$. For inner nodes, it refers to a node sub-range
	$N[o : o + s]$ within the array $N$.

	Thus, all child nodes $N_c$ of node $N_i$ must appear
	consecutively in $N$; i.e., $N[o : o + s]$.
\end{definition}

\begin{figure}
\resizebox{\linewidth}{!}{
\begin{tikzpicture}[
  id/.style={minimum width=1cm, minimum height=1cm, font=\small},
  box/.style={fill=gray!15,draw, minimum width=1cm, minimum height=1cm, font=\small},
  boxE/.style={fill=gray!30,draw, minimum width=1cm, minimum height=1cm, font=\small},
  boxName/.style={minimum width=1cm, minimum height=1cm},
  arr/.style={->, thick},
  background rectangle/.style={fill=gray!5, rounded corners=5pt},
  show background rectangle
]


\node[boxName] (N) {$N:$};
\node[box, anchor=west] (N1) at (N.east) {$(o_i, s_i)$};
\node[box, anchor=west] (N2) at (N1.east) {$(o_{i+1,} s_{i+1})$};
\node[boxE, anchor=west] (N3) at (N2.east) {$\cdots$};
\node[box, anchor=west] (N4) at (N3.east) {$(o_{o_i}, s_{s_i})$};
\node[boxE, anchor=west] (N5) at (N4.east) {$\cdots$};
\node[box, anchor=west] (N6) at (N5.east) {$(o_{k}, s_{k})$};
\node[boxE, anchor=west] (N7) at (N6.east) {$\cdots$};

\node[anchor=north] at (N1.south) {$i$};
\node[anchor=north](name1) at (N2.south) {$i+1$};
\node[anchor=north] at (N3.south) {$\cdots$};
\node[anchor=north] at (N4.south) {$o_i$};
\node[anchor=north] at (N5.south) {$\cdots$};
\node[anchor=north] at (N6.south) {$o_i + s_i$};
\node[anchor=north] at (N7.south) {$\cdots$};




\draw[decorate,decoration={brace,amplitude=8pt}, thick]
  (N4.north west) -- (N6.north east)
  node(size0)[midway, above=10pt] {\scriptsize$N[o_i : o_i + s_i]$};


\draw[thick, shorten <=5pt] 
  (N1) -- ($(N1 |- size0)$);

\draw[->, thick] 
  ($(N1 |- size0)$) 
  -- (size0)
  node[midway, above] {\small children};





\node[boxName] (I) at ($(N.south) + (0, -2)$) {$I:$};
\node[box, anchor=west] (I1) at (I.east) {$I_j$};
\node[box, anchor=west] (I2) at (I1.east) {$I_{j+1}$};
\node[boxE, anchor=west] (I3) at (I2.east) {$\cdots$};
\node[box, anchor=west] (I4) at (I3.east) {$I_{o_{i+1}}$};
\node[box, anchor=west] (I5) at (I4.east) {$I_{o_{i+2}}$};
\node[boxE, anchor=west] (I6) at (I5.east) {$\cdots$};
\node[box, anchor=west] (I7) at (I6.east) {$I_{o_{i+1} + s_{i+1}}$};
\node[boxE, anchor=west] (I8) at (I7.east) {$\cdots$};

\draw[decorate,decoration={brace,amplitude=8pt}, thick]
  (I4.north west) -- (I7.north east)
  node(size1)[midway, above=10pt] {\scriptsize$I[o_{i+1} : o_{i+1} + s_{i+1}]$};


\draw[thick, shorten <=2.5pt] 
  (name1) -- ($(name1 |- size1)$);

\draw[->, thick] 
  ($(name1 |- size1)$) 
  -- (size1)
  node[midway, above] {\small children};
\end{tikzpicture}
}
\caption{The node ($N$) and ID ($I$) arrays of tf::tree.
The \emph{data} array $(o, s)$ specifies subranges within
these arrays. The node $N_i$ is an inner node with children
$N[o_i : o_i + s_i]$, while $N_{i+1}$ is a leaf node with
children $I[o_{i+1} : o_{i+1} + s_{i+1}]$. All children are
contiguous.}\label{fig:tree}
\end{figure}



When interacting with the tree, all children of a node
are always accessed together—either to decide whether
to descend into them (in the case of an inner node)
or to process the corresponding primitives $P_i$
(for a leaf node). This access pattern reinforces our
data layout strategy, which adheres closely to locality
requirements.
Fig.~\ref{fig:tree} illustrates how nodes and ID ranges
are stored and referenced in memory. These two arrays—one
representing the spatial hierarchy, the other indexing
primitives—form the foundation of our data structure.

\begin{definition}[tf::tree]\label{def:tree}
	A \texttt{tf::tree} is defined as the pair $\langle I, N \rangle$,
	where $I$ is an ID array (Def.~\ref{def:ids}) and $N$ is an array
	of nodes (Def.~\ref{def:node}). In a balanced tree, we
	additionally specify the number of children of inner-nodes
	$k\in\mathbb{N}$ and the maximum number of children in an
	inner node $l\in\mathbb{N}$.
\end{definition}

Together, these arrays encode both the structure and contents
of the tree, enabling efficient construction, traversal and query.


\begin{algorithm}[!t]
\SetAlgoLined
\KwIn{%
    AABBs of primitives: $\mathcal{B}$\;
    ID sub-range: $I'$\;
    Node index: $n$\;
    Configuration: $c$
}
\KwOut{%
    Updated node array $N$\;
    Partitioned ID array $I$
}

Compute bounding box: $A_n \gets \bigcup_{i \in I'} \mathcal{B}[i]$\;

\If{$|I'| \leq c.\mathrm{leaf\_size}$}{
    Mark $N[n]$ as a \textbf{leaf} with children range $I^\prime$\;
    \Return
}

Set split axis: $a \gets \arg\max \mathrm{diag}(A_n)$\;
Set $N[n].\mathrm{axis} \gets a$\;
\hspace{1em} $first\_child \gets \mathrm{first\_child\_index}(n)$\;

Partition $I^\prime$ into $c.\mathrm{inner\_size}$ parts using recursive applicatios of \texttt{F::partition} on split axis,
and call \texttt{build\_tree} on each resulting range. Each range
is a child node, starting at index $first\_child$.

Mark $N[n]$ as an \textbf{inner node} with children range
$(\mathrm{first\_child}, c.\mathrm{inner\_size})$ in $N$\;

\caption{\texttt{build\_tree}}
\label{alg:build-tree}
\end{algorithm}



\subsection{Tree Construction}

The construction of spatial partitioning trees generally
follows a common pattern: the current node is divided into
$k$ partitions, and recursion proceeds independently into
each of them as they become child nodes. Because these
child nodes are independent, the recursion can be
parallelized. Thus, the computational bottleneck shifts to
the partitioning step -- our primary area of contribution
with tree construction.

\subsubsection{Partitioning}

Partitioning, in practical terms, means reordering the ID
range $I[n_0:n_k]$ (see Def.~\ref{def:ids}) associated with a
node’s primitives, and dividing it into $k$ non-overlapping
sub-ranges:
\[
	\{I[n_0:n_1],\ I[n_1:n_2],\ \dots,\ I[n_{k-1}:n_k]\}
\]
Each of these sub-ranges corresponds to one child of an
inner node (Def.~\ref{def:tree}), and together they fully
cover $I[n_0:n_k]$.

When a balanced tree is desired, one often turns to
techniques in the spirit of Sort-Tile-Recursive (STR) trees
\cite{str-tree}, where primitives are sorted by their AABB
centroids before the range is evenly partitioned.

To improve upon this and achieve $\mathcal{O}(n\log n)$
construction time (on average), we leverage the insight
that elements $I_i^{n_a}$ and $I_j^{n_b}$ from different
sub-ranges must be separated, but elements within a
sub-range need not be sorted. The same insight is commonly
used in binning-based partitioning approaches \cite{bvh-survey},
though they do not guarantee balanced splits.

The requirement, then, becomes: given a position $n$ in
the range $I$, reorder the elements such that
$\forall_{j<n}I[j] \le I[n]$ and $\forall_{j>n}I[j] \ge I[n]$.
That is, $I[n]$ is the partitioning $n$th element.

\paragraph*{\texttt{F::partition}} is a function in the
that satisfies this requirement
with average-case complexity $\mathcal{O}(n)$. Our default is
\texttt{nth\_element} from the
$\cpp$ standard library \cite{nth-element}.
It is typically implemented using a
combination of \texttt{Quickselect} \cite{quickselect0} and
\texttt{Introselect} \cite{introselect0}. A major advantage
of using this general-purpose algorithm is that it benefits
from continued optimization and tuning
\cite{quickselect1, introselect1, selection}.

In practice, we use \texttt{F::partition} to partition
the ID range $I$ into $k$ disjoint sub-ranges that satisfy
the partitioning requirement. This is achieved through
$k - 1$ applications of the function, using the center of
the primitives’ AABBs along the axis of the node being
partitioned with the largest spatial extent.
We compare the usage of state-of-the-art selection
algorithms at this task in
Section~\ref{sec:eval-partitions}.

\subsubsection{Implementation}

The construction procedure is detailed in
Algorithm~\ref{alg:build-tree}. Before invoking the
algorithm, we initialize the ID array $I$ with the sequence
$[0 : P]$, where $P$ is the number of primitives. The node
array $N$ is then allocated to accommodate the structure of
a perfectly balanced \texttt{k}-ary tree. The index of the
first child of node $n$ is computed as
\[
	\mathrm{first\_child\_index}(n) = k \cdot n + 1,
\]
where $k$ is the fixed number of children for each
inner node. Hence, the nodes are laid out in memory
in such a manner, that the children of a node are
contiguous. Furthermore, all nodes on a specific level
of the tree are contiguous.
Since child nodes are independent, each
recursive dispatch can be performed in parallel,
using \texttt{tbb}\cite{tbb}.

\begin{theorem}
	Algorithm~\ref{alg:build-tree} runs in
	$\mathcal{O}(n \log n)$ time on average, and
	$\mathcal{O}(n \log^2 n)$ in the worst case.
\end{theorem}

\begin{proof}
	This follows from the Master Theorem
	\cite{master-theorem}, based on the complexity guarantees
	of \texttt{F::partition}.
\end{proof}

\paragraph*{API.} Tree construction is performed via 
\texttt{tf::tree::build} method that accepts three parameters:
a range of input objects, an AABB generator, and
a configuration object. The range can be any object 
satisfying the \cpp range concept. The second parameter,
\texttt{make\_aabb}, is a callable that receives an object
from the range and returns its \texttt{tf::aabb}. The third parameter
is a \texttt{tf::tree\_config} struct specifying inner-size and leaf-size.
Optionally it can be passed an additional first parameter, i.e. \texttt{build(F,...)},
where \texttt{F} specifies how the partitioning step is performed via
a static \texttt{F::partition} method. \texttt{F} defaults to
using \texttt{nth\_element}. We provide implementations of
various state-of-the-art algorithms\cite{mini-select}
(see Section~\ref{sec:eval-partitions})
for comparison in the namespace \texttt{tf::strategy}.

\subsection{Search Queries}

Search queries operate by traversing the tree and descending
into nodes whose AABBs contain or intersect the query volume.
Queries may originate from a single primitive or from another
spatial tree or itself.

\subsubsection{Query by Primitive}
This is the simplest case. Large regions of space can be
quickly discarded by checking AABBs. Our implementation avoids
recursion and instead uses a static stack (via
\texttt{small\_vector}~\cite{llvm-small-vector} for safety) to
improve both performance and cache locality.

Such queries are useful for a variety of tasks, including
collecting all primitives that satisfy a predicate or
performing early-exit collision checks at the primitive level.

\paragraph*{API.}
A query by primitive is performed using the function
\texttt{tf::search}, which accepts three parameters:
the \texttt{tf::tree} to be queried, a callable
\texttt{check\_aabb}: $\mathrm{tf::aabb} \to \texttt{bool}$
that filters internal nodes based on their AABB,
and \texttt{apply\_f}: $\mathrm{id}\in I \to \texttt{bool}$,
which is invoked on the primitive id in the leaf nodes.
If \texttt{apply\_f} returns \texttt{true}, the traversal
terminates early. For example, in a collection query,
\texttt{apply\_f} would return \texttt{false} after
appending the intersecting results. We additionally
supply \texttt{tf::search\_broad}, where
\texttt{apply\_f}: $I[n_i:n_j] \to \texttt{bool}$
is invoked on the sub-ID range within leaf nodes.

	{\small 		\begin{algorithm}[!t]
			\SetAlgoLined
			\KwIn{Node $N_0$, Node $N_1$, Functions \texttt{fs}}

			\If{\texttt{fs.abort\_f()} or !\texttt{fs.check\_aabbs}$(N_0.\mathrm{aabb}, N_1.\mathrm{aabb})$}{
				\Return;
			}

			\If{$N_0$ and $N_1$ are both leaves}{
				\texttt{fs.apply\_f}$(N_0.\mathrm{ids}(I), N_1.\mathrm{ids}(I))$\;
			}
			\ElseIf{$N_0$ is a leaf}{
				\ForEach{child $N_i$ of $N_1$}{
					Dispatch \texttt{Search}$(N_0, N_i, \texttt{fs})$\;
				}
			}
			\ElseIf{$N_1$ is a leaf}{
				\ForEach{child $N_i$ of $N_0$}{
					Dispatch \texttt{Search}$(N_i, N_1, \texttt{fs})$\;
				}
			}
			\Else{
				\ForEach{child $N_i$ of $N_0$ and $N_j$ of $N_1$}{
					Dispatch \texttt{Search}$(N_i, N_j, \texttt{fs})$\;
				}
			}

			\caption{\texttt{Search}($N_0$, $N_1$, \texttt{fs})}
			\label{alg:bvh-traversal}
		\end{algorithm}
}

\subsubsection{Query by Tree}
In this case, the goal is to find all overlapping primitive
pairs between two trees. The traversal resembles a nested
search: for each pair of overlapping nodes $(N_0, N_1)$,
we recurse into their children until both are leaf nodes.
At that point, we process the candidate primitive pairs.
All child dispatches can be executed in parallel.
This type of query follows the recursive structure shown in
Algorithm~\ref{alg:bvh-traversal}. 

\paragraph*{API.}
A query by tree is performed using the function
\texttt{tf::search}, which accepts four parameters:
a pair of \texttt{tf::tree}s, a callable
\texttt{check\_aabb}: $(\mathrm{tf::aabb}, \mathrm{tf::aabb}) \to \texttt{bool}$
that filters internal nodes based on their AABBs,
and \texttt{apply\_f}: $(I_0[n_i:n_j], I_1[m_k:m_l]) \to \texttt{bool}$,
which is invoked on the pair of sub-ID ranges within leaf nodes.
If \texttt{apply\_f} returns \texttt{true}, the traversal
terminates early. For example, in a collection query,
\texttt{apply\_f} would return \texttt{false} after
appending the intersecting results. We additionally
supply \texttt{tf::search\_broad}, where
\texttt{apply\_f}: $(I_0[n_i:n_j], I_1[m_k:m_l]) \to \texttt{bool}$
is invoked on the sub-ID range within leaf nodes.

Note that \texttt{apply\_f} must be thread-safe if it performs
any shared-state mutation, such as populating a global list
of collisions.

\subsubsection{Query by Self}
\texttt{tf::tree} supports queries by self via the
\texttt{tf::search\_self} and 
\texttt{tf::search\_self\_broad} functions. Their API
mirrors that of Query by Tree.
This feature is useful for detecting self-intersections
in meshes and merging primitives in $\varepsilon$-proximity.


\begin{figure}[t]
\resizebox{\linewidth}{!}{
\begin{tikzpicture}[
  id/.style={rounded corners, minimum width=1cm, minimum height=1cm, font=\small},
  box/.style={fill=gray!25,draw, minimum width=1cm, minimum height=1cm, font=\small},
  boxTop/.style={fill=gray!10,draw, minimum width=1cm, minimum height=1cm, font=\small},
  boxE/.style={fill=gray!30,draw, minimum width=1cm, minimum height=1cm, font=\small},
  boxName/.style={minimum width=1cm, minimum height=1cm},
  arr/.style={->, thick},
  background rectangle/.style={fill=gray!5, rounded corners=5pt},
  show background rectangle
]


  \node[boxTop] (N0) {4};
\node[boxTop, anchor=south] (N1) at (N0.north) {3};
\node[boxTop, anchor=south] (N2) at (N1.north) {2};
\node[boxTop, anchor=south] (N3) at (N2.north) {1};



  \node[box, right = 2cm of N0] (M0) {4};
\node[box, anchor=south] (M1) at (M0.north) {3};
\node[box, anchor=south] (M2) at (M1.north) {2};
\node[box, anchor=south] (M3) at (M2.north) {1};
\node[boxTop, anchor=south] (M4) at (M3.north) {6};
\node[boxTop, anchor=south] (M5) at (M4.north) {8};

  \node[box, right = 2cm of M0] (O0) {4};
\node[box, anchor=south] (O1) at (O0.north) {3};
\node[boxTop, anchor=south] (O2) at (O1.north) {8};
\node[boxTop, anchor=south] (O3) at (O2.north) {6};
\node[boxTop, anchor=south] (O4) at (O3.north) {2};
\node[boxTop, anchor=south] (O5) at (O4.north) {1};


\draw[->, thick, shorten <=0.1cm, shorten >=0.1cm]
  (N3) -- node[above]{\small Pushing}(M3); 
\draw[->, thick, shorten <=0.1cm, shorten >=0.1cm]
  (M3) -- node[above]{\small Sorting}(O3); 


\node[fill = gray!10, below=0.5cm of N0, minimum height = 0.5cm, anchor=north, draw,semithick, rounded corners] (nm0) {Initial stack};
\node[fill = gray!10, below=0.5cm of O0,minimum height = 0.5cm, anchor=north, draw, semithick, rounded corners] (nm1) {Next stack};

  \draw[semithick] (nm0) -- (N0){};
  \draw[semithick] (nm1) -- (O0){};
  \draw[->, dashed, semithick, shorten <=0.2cm, shorten >=0.2cm] (nm0) -- node[above]{\small Processing} node[below]{\small the popped element}(nm1){};

\end{tikzpicture}
}
\caption{
Visualization of a Top-K Sorted Stack for a binary tree.
On the left is the initial state of the stack after a pop
operation. During processing, elements 6 and 8 are added.
The top 4 elements (2 newly added and 2 existing) are then
sorted, resulting in the stack for the next iteration.
}\label{fig:stack}
\end{figure}

\subsection{Nearness Queries}

Nearness queries---whether between a tree and a
primitive, or between two trees---typically follow a
standard strategy: nodes (or node pairs) are inserted
into a priority queue, ordered by their distance to the
query primitive, or by the distance between their AABBs
in the tree-to-tree case. This approach allows early
pruning of branches using heuristics such as the minimal
distance and the maximum-of-minimal distances
\cite{knn0, eknn}.

The use of a priority queue guarantees that the closest
nodes are explored first, minimizing the number of checks
required. However, research has shown that on modern
processors, reducing the number of operations does not
always translate into faster execution time
\cite{alexandrescu-sorting}. For instance, linear search
outperforms binary search on small sorted arrays due to
better cache locality and branch prediction.

\subsubsection{Top-K Sorted Stack}\label{sec:top-k}

Our implementation follows the standard strategy,
but replaces the priority queue with a stack. To preserve
ordering where it matters most, we sort the top $k$ elements
of the stack after processing the popped element—
as this processing may have pushed several new elements
onto the stack.

Here, $k$ is defined as the number of elements pushed onto the
stack while processing the popped element, plus the number
of nodes in the current inner-leaf. See Figure~\ref{fig:stack}
for a demonstration.


The Top-K Sorted Stack imposes sufficient ordering on the
traversal of the search space that, even though the nearness
query performs more operations than when using a
priority queue, it achieves better overall performance in
practice. These results are evaluated in
Section~\ref{sec:topk}.

\subsubsection{Implementation}

We provide implementations for both query types:
between a tree and a primitive, and between two trees.
The function \texttt{tf::nearness\_search} supports an
optional first parameter that selects the traversal
strategy: \texttt{top\_k\_sorted} or \texttt{priority\_queue},
both defined in the \texttt{tf::strategy} namespace.
The remaining parameters follow the APIs described below.

\paragraph*{API: Primitive to Tree}
This variant accepts three arguments:
the \texttt{tf::tree} being queried,
a callable \texttt{aabb\_metric\_f}:
$\mathrm{tf::aabb} \to \mathbb{R}$,
which computes the distance to an AABB,
and a callable \texttt{closest\_point\_f}:
$i \in I \to \mathrm{tf::closest\_point}$,
which maps a primitive ID to a closest point result.

\paragraph*{API: Tree to Tree}
This variant accepts four arguments:
a pair of \texttt{tf::tree} instances to be queried,
a callable \texttt{aabb\_metric\_f}:
$(\mathrm{tf::aabb}, \mathrm{tf::aabb}) \to \mathrm{tf::aabb\_metrics}$,
which returns the minimum and min–max distances between
bounding volumes~\cite{eknn},
and a callable \texttt{closest\_point\_f}:
$(i_0 \in I_0, i_1 \in I_1) \to \mathrm{tf::closest\_point\_pair}$,
which maps a pair of primitive IDs to a closest point pair result.

\subsection{Dynamics of \texttt{tf::tree}}

All query APIs operate on user-provided callables,
allowing dynamic behavior to be expressed at the query level.
Tree dynamics can thus be achieved by applying
transformations within the supplied callables.

We provide the utility function \texttt{tf::transformed},
which modifies the tree’s AABBs to reflect a given
transformation while preserving structural consistency.
Likewise, users may apply transformations to primitives
within the relevant callables, enabling queries over
geometry undergoing linear motion or deformation,
without rebuilding the tree.

